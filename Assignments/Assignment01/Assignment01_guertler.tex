\documentclass[
	10pt, % Default font size, values between 10pt-12pt are allowed
]{../fphw}

% Template-specific packages
\usepackage{amsmath}
\usepackage[utf8]{inputenc} % Required for inputting international characters
\usepackage[T1]{fontenc} % Output font encoding for international characters
\usepackage{mathpazo} % Use the Palatino font
\usepackage{graphicx} % Required for including images
\usepackage{booktabs} % Required for better horizontal rules in tables
\usepackage{listings} % Required for insertion of code
\usepackage{enumerate} % To modify the enumerate environment

%----------------------------------------------------------------------------------------
%	ASSIGNMENT INFORMATION
%----------------------------------------------------------------------------------------

\title{Linear Algebra Review} % Assignment title
\author{Fabienne Guertler 12-935-508} % Student name
\date{October 9th, 2019} % Due date
\institute{University of Bern} % Institute or school name
\class{Machine Learning} % Course or class name

\begin{document}
	\maketitle
	\section*{Question 1 (10 points)}
	\begin{problem}
		$S = \{v_1,v_2,...,v_n\}$ be an orthogonal set of non-zero vectors in $R^n$. Show that vectors in S are linearly independent.
	\end{problem}
	\paragraph{Proof:} consider linear combination $c_1v_1 + c_2v_2 + … + c_nv_n = 0$ and show that $c_1 + c_2 + … + c_n = 0$. Take dot product of equation with vector $v_j$.
	\begin{equation*}
		0 = c_1 v_1 v_j + c_2 v_2 v_j + ... + c_i v_i v_j
	\end{equation*}
	Because S is an orthogonal set, $v_i v_j = 0$ for $i \neq j$. So all the terms but the i-th one are zero and thus we get
	\begin{equation*}
		0 = c_i v_i v_i = c_i \| v_i \|^2
	\end{equation*}
	This equation implies $c_i = 0$. We conclude that $c_1 + c_2 + … + c_n = 0$ and the vectors are linearly independent.
	
	\section*{Question 2 (15 points)}
	\begin{problem}
		Given a square matrix $A \in \mathbb{R}^{n \times n} $ and a vector $x \in \mathbb{R}^n$. Show that $x^TAx = x^T(\frac{1}{2}A + \frac{1}{2}A^T)x$.
	\end{problem}
	\paragraph{Solution:} the scalar value $x^TAx$ is called a \textbf{quadratic form}.
	\begin{equation*}
		x^T = \sum_{i=1}^{n}x_i(Ax)_i = \sum_{i=1}^{n}x_i(\sum_{j=1}^{n}A_{ij}x_j) = \sum_{i=1}^{n}\sum_{j=1}^{n}A_{ij}x_ix_j
	\end{equation*}
	Any square matrix $A \in \mathbb{R}^{n \times n} $ can be represented as a sum of a symmetric matrix and an anti-symmetric matrix
	\begin{equation*}
		A = \frac{1}{2}(A+A^T) + \frac{1}{2}(A-A^T) = A_s + A_a
	\end{equation*}
	The quadratic form of a purely anti-symmetric matrix is
	\begin{equation*}
		q = x^TA_ax = (x^TA_ax)^T = x^TA_a^Tx = -x^TA_a^Tx = -q
	\end{equation*}
	Which implies that $q = 0$.\\
	The quadratic form of a square matrix
	\begin{equation*}
		x^TAx = x^T(A_s + A_a)x = x^TA_sx + x^TA_ax
	\end{equation*}
	We know that $x^TA_ax = 0$. Finally we find that
	\begin{equation*}
		x^TAx = x^TA_sx = x^T\frac{1}{2}(A+A^T)x
	\end{equation*}
	
	\section*{Question 3 (15 points)}
	\begin{problem}
		Show that if $(A + B)^{-1}=A^{-1}+B^{-1}$ then $AB^{-1}A=BA^{-1}B$.
	\end{problem}
	\paragraph{Solution:} First, multiply both sides of equation with $A+B$ on the right:
	\begin{align*}
		(A+B)^{-1}(A+B) &= (A^{-1}+B^{-1})(A+B)\\
		I &= I + A^{-1}B + B^{-1}A + I
	\end{align*}
	This leads to
	\begin{equation*}
		A^{-1}B = -B^{-1}A - I \quad\text{and}\quad AB^{-1} = -I - BA^{-1}
	\end{equation*}
	If we multiply $A^{-1}B = -B^{-1}A - I$ with $B$ on the left we get
	\begin{equation*}
		BA^{-1}B = B(-B^{-1}A-I) = -(A + B)		
	\end{equation*}
	If we multiply $AB^{-1} = -I - BA^{-1}$ with $A$ on the right we get
	\begin{equation*}
		AB^{-1}A = (-I - BA^{-1})A = -(A + B)
	\end{equation*}	
	Finally we get
	\begin{equation*}
		BA^{-1}B = -(A+B) = AB^{-1}A
	\end{equation*}
		
	\section*{Question 4 (15 points)}
	\begin{problem}
		Use definition of trace to show that $tr(A+b)=trA + trB$, where $A,B \in \mathbb{R}^{n \times n}$.
	\end{problem}
	\paragraph{Proof:} the $(i,i)-th$ entry of $A+B$ is $a_{ii}+b_{ii}$.
	\begin{align*}
		tr(A+B) &= (a_{11}+b_{11})+(a_{22}+b_{22})+...+(a_{nn}+b_{nn}) \\
		&= (a_{11}+a_{22}...+a_{nn})+(b_{11}+b_{22}...+b_{nn}) \\
		&= trA + trB
	\end{align*}
	
	\section*{Question 5 (15 points)}
	\begin{problem}
	Show that if $(\lambda_i,x_i)$ are the i-th eigenvalue and i-th eigenvector of a non-singular and symmetric matrix $A \in \mathbb{R}^{n \times n}$, then $(\frac{1}{\lambda_i},x_i)$ are the i-th eigenvalue and i-th eigenvector of $A^{-1}$.
	\end{problem}
	\paragraph{Solution:} multiply $Ax_i=\lambda_i x_i$ with $A^{-1}$ on the left
	\begin{align*}
		Ax_i &= \lambda_i x_i \\
		A^{-1}Ax_i &= \lambda_i A^{-1} x_i \\
		x_i &= \lambda_i A^{-1} x_i \\
		\frac{1}{\lambda_i}x_i &= A^{-1}x_i
	\end{align*}
	Thus $\frac{1}{\lambda_i}$ is the i-th eigenvalue and $x_i$ the i-th eigenvector of $A^{-1}$

	\section*{Question 6 (10 points)}
	\begin{problem}
		Show that $rank(A)\leq min\{m,n\}$, where $A \in \mathbb{R}^{m \times n}$.
	\end{problem}

	\section*{Question 7 (20 points)}
	\begin{problem}
		In each of the following cases, state whether the real matrix A is guaranteed to be singular or not. Justify your answer in each
		case.
		\medskip
		\begin{enumerate}[(\itshape a\normalfont)]
			\item $A \in \mathbb{R}^{(n+1) \times n}$ is a full rank matrix.
			\item $|A| = 0$.
			\item $A$ is an orthogonal matrix.
			\item $A$ has no eigenvalue equal to zero.
			\item $A$ is a symmetric matrix with non-negative eigenvalues.
		\end{enumerate}
	\end{problem}

	\paragraph{Solutions:} Inverse of a square matrix $A \in \mathbb{R}^{n \times n}$ is denoted $A^{-1}$ andis the unique matrix such
	that
	\begin{equation*}
		A^{-1}A = I = AA^{-1}
	\end{equation*}
	Matrix A is singular, or non-invertible if the inverse $A^{-1}$ of matrix does not exist.
	
	\begin{enumerate}[(\itshape a\normalfont)]
		\item singular. A is not a square matrix.
		\item singular. $|A| = 0$ if and only if A is singular. If A is singular then it does not have full rank, and hence its columns are linearly dependent. In this case, the set S corresponds to a “flat sheet” within the n-dimensional space and hence has zero volume.
		\item non-singular. A square matrix $A \in \mathbb{R}^{n \times n}$ is orthogonal if all its columns are orthonormal.
		\begin{equation*}
			A^TA = I = AA^T
		\end{equation*}
		We know that $|AB| = |A||B|$. Thus we have $|I| = 1 = |AA^T| = |A||A^T| = |A||A| = |A|^2 $. Because we have $|A|^2 = 1$ every orthogonal matrix has a determinant either 1 or -1. If $|A| \ne 0$ matrix A is invertible, non-singular.
		\item non-singular. The determinant of A is the product of its eigenvalues. So if it has an eigenvalue 0 the determinant would also be 0. On the other hand if matrix has no eigenvalue equal to zero, its determinant is also non-zero and thus it is invertible.
		\item non-singular. Matrix which has only non-negative eigenvalues $\lambda_i \geq 0$ are positive semidefinite. Since this includes eigenvalue 0 its determinant is 0 and thus its singular. If A is positive definite, meaning that $\lambda_i > 0$ its determinant would be positive and thus non-singular.
	\end{enumerate}
	
\end{document}
